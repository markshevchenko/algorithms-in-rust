\chapter{Введение}

Версия \input{../version.txt}.

\section{История}

Пару лет назад в одном из Telegram-чатов мы обсуждали, какая из книг по алгоритмам лучше.
Как всегда, участники не смогли прийти к общему мнению.

Однако во время дискуссии у меня появилась интересная мысль.
Сам я предпочитаю трёхтомник Кнута~--- это фундаментальная книга, где можно регулировать уровень погружения.
Вы можете читать только описания алгоритмов и пропускать доказательства.

Но \textbf{вымышленный язык ассемблера}, на котором реализованы алгоритмы, затрудняет их понимание.
Да, в семидясятые, когда были написаны первые тома, ассемблеры был распространены и большинство программистов могли читать и понимать подобный код.
Но сейчас большинство программистов пишут только на высокоуровневых языках.

Что же вместо ассемблера?~--- задался я вопросом.
Ответ пришёл сразу: только Rust.
Это высокоуровневый язык с сильной типизацией и развитой системой типов.
Его выразительности хватит, чтобы описывать алгоритмы, избегая низкоуровневых подробностей.
С другой стороны, это системный язык, близкий к железу.
В отличие от, например, Python или Java, он позволяет рассуждать о производительности и работе с памятью.
И он компактный, в отличие от С++.
Я бы даже сказал~--- стройный.

К сожалению, такой книги не было. И тогда я решил её написать.

\section{Аудитория}

При написании книги всегда очень важно обрисовать аудиторию. Долгие годы программирование было связано с \textit{прикладной математикой}. Для того, чтобы стать программистов, надо было получить высшее образование.

Но в 1990-е и 2000-е ситуация изменилась. В индустрию пришли специалисты без образования, способные, как оказалось, решать подавляющее большинство практических задач.

Однако современные издательства предлагают им либо учебники, опирающиеся на математику, которую не дают в школе, либо совсем примитивные пособия.

Я большой поклонник понятного обучения~--- стараюсь избегать теоретических материалов, которые не опиратся на практику. Поэтому я буду стараться очерчивать проблему и затем предлагать способы её решения.

С другой стороны, я люблю математику и люблю теорию.
Дав основной материал, я постараюсь закрепить его хорошей базой.

Если какие-то объяснения покажутся вам сложными, вы можете их пропустить.
Или разбораться~--- это может быть даже лучше.
Но сложный текст не должен вас останавливать.
Если не понимаете материал и не можете разобраться, переходите к следующей главе.

Сам материал составлен так, чтобы его можно было читать последовательно.
Вторая глава может рассказывать о чём-то таком, что пригодится в пятой главе, но не наоборот.

Последний принцип, как бы он ни был правилен, на практике недостижим. Всё равно встречаются связанные знания, поэтому, рассказывая о чём то одном, приходиться забегать вперёд и рассказывать о другом.

В этим случаях я буду явно обозначать ситуацию и ссылаться на главу, где все ветки сойдутся воедино.

\section{Версии}

\section{Пожертвования}
